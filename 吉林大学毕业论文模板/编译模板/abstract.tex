\vspace*{-8mm}
%% “摘要”二字中间空两格、小二号字、黑体、居中
\begin{center}
	\zihao{-2} {\heiti{摘\ \ \ 要:}}
\end{center}
\renewcommand\baselinestretch{1.5} 

%% 摘要内容用四号字、宋体, “关键词”三字加粗,各关键词之间要有空格
{\zihao{4}时间并行算法中的一个重要类别是Parareal算法, 它广泛应用于解决演化方程. 为了实现有效的加速, 算法中选择合适的粗网格传播器至关重要. 在这项工作中, 我们探讨了学习型粗网格传播器的使用. 基于误差估计框架, 我们提出了一种构建具有稳定性和相容精度的粗网格传播器的方法. 此外, 我们为所得Parareal算法提供了初步的数学保证. 在各种设置上的数值实验, 例如线性扩散模型、Allen-Cahn模型和粘性Burgers模型, 表明与一些传统且广泛使用的粗网格传播器相比, 学习型粗网格传播器可以显著提高并行效率. 

\vskip5mm

%% “关键词”三字加粗,各关键词之间要有空格
{\textbf{关键词:}} 椭圆问题; 单步法; 收敛方法; 机器学习}
